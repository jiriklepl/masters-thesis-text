
\chapwithtoc{Conclusion}

In this thesis, we have described an extension of the deferred type inference algorithm with constrained version of subtyping (described in \cref{defer_solve,sec:typesystem}), and applied it to inference of types of an extended variant of the \cmm language, which is described in \cref{chap2}. Our extended \cmm language features polymorphic types with overloading based on multi-parameter typeclass with user-specified functional dependencies, record types, and a type-driven form of automatic resource management. Overall, this fulfills the main goals of the thesis.

The secondary output of the thesis, a prototype compiler for the language, translates the extended \cmm language into lower-level, simpler LLVM assembly code, which can be further used to produce executables and libraries. The compilation process and each of its phases are detailed in \cref{chap3}.

In \cref{chap4}, we discuss possible practical use-cases of the explored and implemented language features. The discussion is accompanied by many code examples, demonstrational programs and a brief comparison with similarly-focused languages. More examples, used as tests during the development of the compiler, are present in the supplementary repository\cmmrepo.

\section*{Future work}

Here, we list several possible extensions to our work that did not fit the scope of the thesis, but would be interesting to explore nevertheless:

\begin{description}
    \item[Existential types] We based our solution for type inference on deferred solving algorithm capable of solving existential types \cite{vytiniotis2011outsidein}. One natural course of future development could be trying to combine these with our extensions to the language. Utility of existential types in imperative languages has been demonstrated for example by \citet{grossman2002existential}.

    \item[Stronger subtyping system] We discussed some inefficiencies of the type system we developed. For future work, we leave a revised approach that constitutes two constraint generation phases capable of generating stronger subtype constraints, as described in \cref{sec:weakness}.

    At this point, it is unclear whether the complications induced by better subtype support, mainly the complexity of inference, would be sufficiently counter-weighted by the newly gained features. Further research is needed to show if there is any significant benefit in letting the programmers to interact with the subtype system -- for example, defining their own semantics and interpretation for constness-like annotations.
\end{description}
